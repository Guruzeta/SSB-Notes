\documentclass[12pt]{article}
\usepackage[utf8]{inputenc}
\usepackage{physics}
\renewcommand{\familydefault}{\sfdefault}
\usepackage[a4paper, left=1in, right=1in, top=1.5in, bottom=1in]{geometry}
\title{Notes on Spontaneous Symmetry Breaking}
\author{Guru Kalyan Jayasingh}
\date{June 2020}

\begin{document}

\maketitle

\section{Broad Overview of Symmetry Breaking}
Definition:

When the state $\ket{\Psi}$ isn't invariant under the symmetry operator $\hat{U}$, 
s.t. $\comm{\hat{H}}{U}=0$\\(i.e. it's a symmetry of the system), then the *\textbf{state}* is said to have spontaneously broken the symmetry.\\
\\
Observation 1: For every such $\ket{\Psi}$, there exists a multitude of states $\ket{\Phi}$ s.t. they're degenerate. 
One can generate the set {$\ket{\Phi}$} by the rule $\ket{\Phi}\;=\;\hat{U}\ket{\Psi}$.\\
Check that $E_{\ket{\Psi}}\;=\;E_{\ket{\Phi}}$.\\
\\
Order Parameter Operator: For the set of *broken* symmetry states, one can define an order parameter operator $\hat{O}$.
The action of this operator is defined as:
\begin{itemize}
    \item 
        Each of the symmetry related states {$\ket{\Phi}$} are the eigenstates of this operator with distinct non-zero eigenvalues.
    \item The expectation value $\hat{O}$ in a symmetric state is 0. 
\end{itemize}
In general $\comm{\hat{O}}{\hat{H}}\neq 0$ (however there are *cases* where it happens to be true as well).\\
Observation 2: For the general case, we can see that $\ket{\Psi}$ will not be an Energy eigenstate. Moreover, it cannot be a thermal mixture of energy eigenstates.\\
Proof: Taking $\comm{\hat{O}}{\hat{H}}\neq 0$, it can't be an eigenket since that would imply $\hat{H}$ and $\hat{O}$ commute.
For the thermal mixture case, let 
$$\ket{\Psi}\;=\;\sum e^{-\beta E_n }\ket{n}$$
$$\implies \hat{H}\ket{\Psi}\;=\;\pdv{\ket{\Psi}}{\beta}$$
$$\implies \hat{O}\hat{H}\ket{\Psi}\;=\; \hat{O}\pdv{\ket{\Psi}}{\beta}\;=\;\pdv{\hat{O}\ket{\Psi}}{\beta}$$
$$\implies \hat{O}\hat{H}\ket{\Psi}\;= O \pdv{\ket{\Psi}}{\beta} $$
$$\implies \comm{\hat{H}}{\hat{O}}=0$$
\\
Conclusion: Clearly the symmetry broken state $\ket{\Psi}$ isn't in a thermal equilibrium! Then shouldn't these states be unstable? 
No. We do get to observe them in day-day life and these clearly unequivocally exist with large lifetimes(and hence are stable in the colloquial sense).
How does this happen then? Ans - Due to the large singularity of thermodynamic limit. \\
At Thermodynamic limit( $N\rightarrow \infty$, $V\rightarrow \infty$ with $\displaystyle{\frac{N}{V}=\; constant}$) :
\begin{itemize}
    \item  $<\comm{\hat{H}}{\hat{O}}>\;=\;0$
    \item SSB states become orthogonal to each other i.e. $\braket{\Phi}{\Psi}=0$
    \item SSB states become degenerate with Symmetric energy eigenstates, thus becoming eigenstates themselves. Hence they can occur in thermal equilibrium.
\end{itemize}
Thermodynamic limit, being qualitatively different as above from large finite N,V makes the limit itself a singular one(defined below). Ergo - Thermodynamic limit is an idealisation, only to serve as a guide and not a real description.\\
\\
Observation 3: Symmetric Hamiltonians (by symmetry here a global symmetry is implied) exhibit the following properties:

\begin{itemize}
    \item The decompostion of $\hat{H}\;=\;\hat{H_0} + \sum_{\Vec{K}}\hat{H}_{\Vec{K}}$ is possible, where the 1st part realizes $\Vec{k}=0$ part of the F.T. i.e. Centre of Mass part and the 2nd part describes internal degrees of freedom(basically a fourier decomposition into different non-zero and zero $\Vec{k}$ parts).
    \item $\comm{\hat{H}_0}{\sum_{\Vec{K}}\hat{H}_{\Vec{K}}}\;=\;0$
\end{itemize}

\end{document}
